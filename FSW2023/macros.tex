
\newcommand{\newterm}[1]{\emph{#1}}    % For newly introduced terms.
\newcommand{\code}[1]{\texttt{#1}}     % Used for bits of Ada inline.
\newcommand{\filename}[1]{\texttt{#1}} % For file names.
\newcommand{\SPARK}{\textsc{Spark}}    % For ease of small capping SPARK


% Redefine the underscore command so latex will break words at underscores. Found this trick at
% http://mrtextminer.wordpress.com/2009/02/26/break-a-long-word-containing-underscores-in-latex/
%
\let\underscore\_
\newcommand{\breakingunderscore}{\renewcommand{\_}{\underscore\hspace{0pt}}}

% Issue the changed underscore command to the whole document.
\breakingunderscore


% The \note macros are useful for creating easy to see notes.

% Carl Brandon
\long\def\cnote#1{\marginpar{CB}{\small \ \ $\langle\langle\langle$\
{#1}\
    $\rangle\rangle\rangle$\ \ }} 

% Peter Chapin
\long\def\pnote#1{\marginpar{PC}{\small \ \ $\langle\langle\langle$\
{#1}\
    $\rangle\rangle\rangle$\ \ }} 

% Sevan Golnazarian
\long\def\snote#1{\marginpar{SK}{\small \ \ $\langle\langle\langle$\
{#1}\
    $\rangle\rangle\rangle$\ \ }} 


%% The following are settings for the listings package. See the listings package documentation
%% for more information.
%%
%\lstset{language=C,
%        basicstyle=\small,
%        stringstyle=\ttfamily,
%        commentstyle=\ttfamily,
%        xleftmargin=0.25in,
%        showstringspaces=false}

% Define listing parameters for Ada 2012. Base it on Ada 2005.
\lstdefinelanguage{Ada2012}[2005]{Ada}
{
  morekeywords={some},
  sensitive=false,
  breaklines=false,
  showstringspaces=false,
  xleftmargin=0.25in,
  basicstyle=\small\sffamily,
  columns=flexible,
}
\lstset{language=Ada2012, showlines=true}

% Wide equivalent of the listings package lstset. The parameter is the amount to indent the code
% (may be negative).
%
\lstnewenvironment{widelisting}[1]
   {\lstset{language=Ada2012,xleftmargin=#1}}
   {}
   
% Wide listing from a file using the Listings package lstinputlisting. First parameter is
% path name to file (can be relative). Second parameter is the amount to indent the code (may
% be negative).
%
\newcommand{\wideinputlisting}[2]{%
   \lstinputlisting[language=Ada2012,xleftmargin=#2]{#1}}


% An environment for displaying use cases.
%   This environment takes three parameters:
%   \param #1: The name of the use case.
%   \param #2: The actor who participates in the use case.
%   \param #3: The context in which the use case executes.
%
%   The body of the environment is the action associated with the use case.
\newsavebox{\UseCaseName}    % Create some boxes to hold the necessary text.
\newsavebox{\UseCaseActor}   % We need to do this because we can't use the
\newsavebox{\UseCaseContext} % environment parameters in the 'end' definition.
\newsavebox{\UseCaseAction}
\newenvironment{usecase}[3]
 {
  \sbox{\UseCaseName}{\bfseries #1}  % The name is easy.
  \sbox{\UseCaseActor}{#2}           % The actor is easy.
  \begin{lrbox}{\UseCaseContext}     % Format the context in a minipage.
    \begin{minipage}{3.25in}
    #3
    \end{minipage}
  \end{lrbox}
  \begin{lrbox}{\UseCaseAction}      % The environment body becomes the action.
  \begin{minipage}{3.25in}}
 {\end{minipage}
  \end{lrbox}
\begin{center}   % Now spew forth the table using the information collected above.
\begin{tabular}{|l||p{3.5in}|} \hline
\multicolumn{2}{|c|}{\usebox{\UseCaseName}} \\ \hline
Actor   & \usebox{\UseCaseActor}   \\ \hline
Context & \usebox{\UseCaseContext} \\ \hline
Action  & \usebox{\UseCaseAction}  \\ \hline
\end{tabular}
\end{center}}
