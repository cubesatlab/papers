%%%%%%%%%%%%%%%%%%%%%%%%%%%%%%%%%%%%%%%%%%%%%%%%%%%%%%%%%%%%%%%%%%%%%%%%%%%%
% FILE    : main.tex
% AUTHOR  : (C) Copyright 2015 by Vermont Technical College
% SUBJECT : Potential paper on CubedOS
%
% TODO:
%
% Send comments or bug reports to:
%
%    Carl Brandon
%    Vermont Technical College
%    Randolph Center, VT 05061
%    carl.brandon@vtc.edu
%%%%%%%%%%%%%%%%%%%%%%%%%%%%%%%%%%%%%%%%%%%%%%%%%%%%%%%%%%%%%%%%%%%%%%%%%%%%

%+++++++++++++++++++++++++++++++++
% Preamble and global declarations
%+++++++++++++++++++++++++++++++++
\documentclass[twocolumn]{article}
\usepackage{authblk}
\usepackage[pdftex]{graphicx}
\usepackage{listings}
\usepackage{hyperref}
\usepackage{url}

\lstdefinelanguage{Ada2012}    %Define listing parameters for Ada 2012
   [2005]{Ada}                 %... base it on Ada 2005
   {morekeywords={some},
    sensitive=false,
    breaklines=false,
    showstringspaces=false,
    basicstyle=\small\sffamily,
    columns=flexible,
    }
\lstset{language=Ada2012, showlines=true}

\newcommand{\SPARK}{\textsc{Spark}}   % for ease of small capping SPARK

% Stock article class title page formatting...
%%%%%%%%%%%%%%%%%%%%%%%%%%%%%%%%%%%%%%%%%%%%%%
\title{HIGH INTEGRITY SOFTWARE FOR CUBESATS AND OTHER SPACE MISSIONS}
\author[*]{Carl Brandon}
\author[**]{Peter Chapin}
\affil[*]{Vermont Technical College, Science Department}
\affil[**]{Vermont Technical College, Computer Information Systems Department}
\renewcommand\Authands{ and }
\date{}

%++++++++++++++++++++
% The document itself
%++++++++++++++++++++
\begin{document}

\twocolumn[
  \begin{@twocolumnfalse}
    \maketitle
\begin{abstract}
  % Summarize the contents of the paper using at least 70 and at most 400 words.

  We currently have an operating single CubeSat launched as part of NASA's ELaNa IV program on
  November 19, 2013, the first satellite of any kind launched by a college or university in New
  England. Many CubeSat failures have been attributed to software failures. Of the twelve
  university CubeSats that were launched with ours, we are the only one that is fully
  functional. One damaged their batteries the first day due to a software error and worked
  partially for a short time only in sunlight, one had partial contact for a week, one lasted
  four months, and eight were never heard from. These other CubeSats primarily used the C
  language. We are using the most reliable software technology ever sent into space. We used the
  \SPARK\ 2005 Toolset and Ada language in the construction of our software. Ada is used in
  almost all European Space Agency and many NASA rockets and spacecraft, and in most European
  rail systems and nuclear power plants. \SPARK\ is used in commercial aviation (Rolls-Royce
  Trent jet engines, ARINC ACAMS system, Lockheed Martin C130J), military aviation (EuroFighter
  Typhoon, Harrier GR9, AerMacchi M346), air-traffic management (UK NATS iFACTS system), rail
  (numerous signaling applications), and medical (LifeFlow ventricular assist device)
  applications.

  We are using \SPARK/Ada, with its reduction of errors by a factor of about 100 compared with
  C. \SPARK\ is a formally defined programming language and a set of verification tools
  specifically designed to support the development of high integrity software, and can formally
  verify information flow, freedom from runtime errors, functional correctness, and security and
  safety policies.

  Ours is the first spacecraft to use \SPARK. We are currently upgrading our CubeSat software to
  \SPARK\ 2014, and will then work on improving some of the algorithms in that software. We
  would then have a very reliable software platform, CubedOS, that other projects could use a a
  base for their CubeSat or other spacecraft projects. We have also been converting NASA
  Goddard's GPS Enhanced Onboard Navigation System (GEONS) to \SPARK, and have discovered and
  fixed errors by this process. Our next CubeSat, Lunar IceCube with Morehead State University
  (PI), Goddard Space Flight Center (BIRCHES \& Lunar transfer trajectory)and the Jet Propulsion
  Lab (Iris 2) is self propelled with a Busek iodine ion drive which will go to the Moon on the
  Space Launch System EM-1 flight in 2018. This software will be much more complex, dealing with
  power management, the ADACS, the infrared spectrometer (BIRCHES), the data and navigation
  radio (Iris 2), the electrical power system, and aim the photo voltaic panels and ion
  thruster. The software will carry out the navigation plan and deal with ground based commands
  and upgrades. \SPARK’s reliability will be necessary for this.

  %\keywords{\SPARK, student project, CubeSat}
\end{abstract}
  \end{@twocolumnfalse}
]

%\begin{enumerate}
%\item Vermont Lunar CubeSat Description
%  \begin{enumerate}
%  \item Hardware
%  \item Software
%    \begin{enumerate}
%      \item Software Metrics
%    \end{enumerate}
%  \item Mission
%  \item ELaNa IV Results
%  \end{enumerate}
%
%\item \SPARK\ 2014 Description and Characteristics
%\begin{enumerate}
%  \item Toolset
%  \item Language
%  \item Where \SPARK\ would have helped
%    \begin{enumerate}
%    \item Ariane V
%    \item Boeing 787
%    \end{enumerate}
%  \end{enumerate}
%
%\item New Work
%\begin{enumerate}
%  \item CubedOS
%  \item GEONS translation
%  \item Lunar IceCube
%    \begin{enumerate}
%    \item Collaborators
%    \item Spacecraft description
%      \begin{enumerate}
%      \item Iodine ion drive
%      \item Computer
%      \item BIRCHES
%      \item Iris 2
%      \item ADAC
%      \item PV panels
%      \end{enumerate}
%    \item Mission
%      \begin{enumerate}
%      \item Transfer to lunar orbit
%      \item Collecting data
%      \item Data download
%      \item End - Moon or Mars?
%      \end{enumerate}
%    \end{enumerate}
%  \end{enumerate}
%\end{enumerate}


\section{Introduction}
\label{sec:introduction}

CubeSats are small (10\,cm x 10\,cm x 10\,cm) spacecraft that are in a size class often referred
to as nano-satellites. Typical CubeSat missions are executed in low Earth orbit. They are placed
there with the help of much larger launch vehicles that are launching full sized satellites for
some unrelated purpose (e.g., satellite television). In recent years interest is rising in using
CubeSats on more ambitious missions such as accompanying deep space spacecraft to the moon and
planets. Deep space CubeSat-only missions are also being contemplated. \pnote{We need some
  citations here!}

As with all spacecraft, software plays a critical role in the success of CubeSat missions.
Unfortunately many CubeSat missions have been plagued with catastrophic software problems
\pnote{More citations, please}. Historically CubeSats have been created in a relatively informal
fashion by organizations, such as universities, with very limited budgets and inexperienced
personnel, such as students. Commonly the emphasis of the development effort is on the physical
and electrical engineering of the spacecraft with software development given only incidental
attention. The result is often inadequate, faulty software.

As CubeSat missions grow even more ambitious, software complexity and the requirements for
software reliability grow accordingly. Because of limited physical space and other resources, as
well as limited development and support personnel, CubeSat missions often must dispense with the
levels of redundancy and review afforded to high profile, multi-billion dollar NASA/ESA
missions. This further exacerbates the problems with creating robust software for CubeSats.

At Vermont Technical College's CubeSat Laboratory we are working to address these issues. We are
developing a general purpose, open source application framework for CubeSat flight software that
we call \newterm{CubedOS}. While not precisely an operating system (it is intended to run on top
of an existing operating system or runtime environment), CubedOS does provide some basic
infrastructure and supporting services of interest to flight software developers. A well
documented, reusable system such as CubedOS will simplify the construction of flight software
and improve its reliability by cutting down on the amount of supporting code that must be
developed by each mission.

NASA's Goddard Spacecraft Flight Center has, in fact, developed a very similar framework called
the Core Flight System (CFS) \pnote{Citation needed}. Unlike CubedOS at this time, CFS is well
established with a large community of users and several active and planned missions
\pnote{Citation needed}. In fact, it is our goal to eventually provide a CubedOS/CFS bridge that
would allow a current CFS-based mission to migrate to CubedOS incrementally if desired.

However, unlike CFS, which is written in unverified C, CubedOS is entirely written in \SPARK\
2014 and verified free of runtime error in the sense meant by \SPARK. \pnote{Citation?} We also
intend to use \SPARK, as time and resources allow, to prove that CubedOS satisfies certain
functional correctness properties as well.

CubedOS applications are organized as a collection of \newterm{modules}. Each module is usually
implemented as a hierarchy of Ada packages with a single root. Every module has an ID number
assigned to that module by the application developer. System modules providing basic services
useful to many, if not most, missions are given ``well known'' ID numbers by a central authority
following the approach used when assigning well known port numbers for TCP based network
services.

Modules communicate with each other via a fairly standard, asynchronous message passing system.
Each module is associated with exactly one \newterm{mailbox} from which it receives all messages
destined to it. Modules send messages by adding the messages to the mailboxes of the recipients.
The mailboxes are implemented as protected objects.

Every module contains at least one task that reads that module's mailbox and processes the
messages received. Modules may contain additional, internal tasks if necessary. Because \SPARK\
limits the number of entries on a protected object to one, the |Send| procedure is implemented
as a protected procedure and not an entry. One consequence of this design is that messages might
be lost of a mailbox fills to capacity. There is a way for a message sender to detect such loss
if desired.

The mailboxes in CubedOS are organized in \newterm{communication domains} consisting of an array
of mailboxes indexed by module ID numbers. CubedOS supports the use of multiple communication
domains identified by domain IDs set by the application developers. Module IDs are scoped by
domain IDs so, for example, two different modules can have the same Module ID as long as they
are in different domains. This architecture anticipates the extension of CubedOS to distributed
systems, such as CubeSat ``swarms'' as envisioned by some mission planners.

Because CubedOS mailboxes are stored in an array, they must all have the same data type.
However, this is problematic because the messages received by the various modules will, in
general, contain different collections of typed parameters. Thus the messages are stored in the
mailboxes in a raw binary format as strings of 8~bit octets. \pnote{Do we need to say something
  about how the request/reply format requires this architecture?} This means the typed message
arguments must be encoded in some way into raw binary data before sending and then decoded by
the receiver back into properly typed message parameters.

This encoding and decoding process closely resembles the marshaling and unmarshaling of
procedure arguments that takes place in the context of a remote procedure call (RPC) protocol.
Although CubedOS has been conceptualized as a message passing system, it could just as easily be
thought of as providing an asynchronous RPC discipline between modules.

Following the tradition of many RPC systems \pnote{Citation needed}, CubedOS allows the
developer to describe the interface of a module in specialized interface definition language. A
tool is then provided, the main subject of this paper, that compiles the interface definition
into packages to perform the necessary encoding and decoding of message parameters. Since we
selected the External Data Representation (XDR) standard \cite{rfc-4506} for the binary format,
we call our tool Merc. The interface definition language we use is an extension of that
described in \pnote{add citation to the RFC} which we call \newterm{modified XDR} (MXDR).

Merc outputs \SPARK\ packages that can be proved free of runtime error. The modifications in
the MXDR language account for the special needs of the \SPARK\ target language. In particular:

\begin{itemize}
\item We extended XDR to allow the declaration of constrained ranges on types. These constraints
  are mapped to \SPARK\ declarations in an obvious way. This was necessary since the use of
  constraints plays an important role in making the code provable.

\item We defined a special kind of structure called a \newterm{message struct} that informs
  Merc when to generate encoding and decoding subprograms. The members of a message struct
  correspond to the parameters of a message. Ordinary structures are mapped to \SPARK\ records
  and, by themselves, do not define messages.

\item We allow a message struct to optionally carry a \newterm{message invariant} that describes
  a condition that must hold on the values inside the message struct. That condition is mapped
  to a precondition on the encoder subprogram (to ensure the parameters of the encoder have the
  invariant), and to a postcondition on the decoder subprogram (to inform the calling
  environment about the invariant).
\end{itemize}

The rest of this paper is organized as follows. Section~2 gives more details about the MXDR
language. Section~3 describes the code generated by Merc. Section~4 discusses the performance
of the generated code and related issues. Section~5 concludes. We make use of a running example
throughout.


\section{Vermont Lunar CubeSat}
\label{sec:vtlunar}

\subsection{Hardware}
\label{sec:vtlunar-hardware}

\subsection{Software}
\label{sec:vtlunar-software}

% TODO: Include some material on the software metrics.

\subsection{Mission}
\label{sec:vtlunar-mission}

\subsection{ELaNa IV Results}
\label{sec:vtlunar-elana}



\section{SPARK 2014}
\label{sec:spark2014}

%\lstMakeShortInline[basicstyle=\ttfamily]|
\lstMakeShortInline|

The current version of the \SPARK\ toolset and language definition is \SPARK\ 2014. It is a
major enhancment over the earlier \SPARK\ 2005 toolset and language definition we briefly
described in Section~\ref{sec:vtlunar-software}. The \SPARK\ 2014 language supports a much
larger subset of Ada, allowing more natural designs. The \SPARK\ 2014 toolset uses more modern
theorem provers, and is more easily extensible to use additional provers as they become
available. The net effect of these enhancements is that \SPARK\ 2014 is much easier to use,
allowing the developer to focus more on the problem being solved and less on working around the
idiosyncrasies of the programming environment.

In this section we give an overview of the \SPARK\ 2014 toolset and language so the reader can
better understand the nature of \SPARK\ programming and the advantages it offers. For a more
complete description of \SPARK\ 2014 see, for example, \cite{McCormick-Chapin-2015}.

All current and future software development done by the CubeSat Laboratory at Vermont Technical
College, including the work described in Section~\ref{sec:new}, is being done using \SPARK\
2014. Unless otherwise stated all following uses of \SPARK\ in this paper refer to \SPARK\ 2014.

\subsection{Toolset}
\label{sec:spark2014-toolset}

The \SPARK\ tools consist of a modified Ada compiler together with a verification condition
generator and one or more back-end theorem provers.

AdaCore's GNAT Ada compiler has been modified to understand the additional \SPARK\ aspects,
described in the next section, and to verify, upon request, conformance to the restrictions of
the \SPARK\ language. Certain diagnostic messages produced by ``the \SPARK\ tools'' are actually
produced by the modified Ada compiler before the specialized tools are run. These are typically
messages related to the structure of the program (i.e., syntax errors in the \SPARK\ specific
constructs).

An additional tool, GNATprove, performs detailed data and information flow analysis, described
in the next section, and generates verification conditions for the provers. Conceptually
GNATprove produces a verification condition, or ``check,'' for every place where the Ada
language mandates a runtime check. If these verfication conditions are proved, or
``discharged,'' it means the runtime check will never fail. Examples of such runtime checks
include: out of bounds array access, arithmetic overflow, division by zero, and some other
things.

In addition the Ada language allows the programmer to express range constraints on values to
ensure the results of computations are always in an appropriate range (e.g., never negative,
always in the range 1 to 100, etc.) Ada normally includes runtime checks to verify these
constraints; GNATprove generates verification conditions that, if discharged, will statically
show they never fail.

Furthermore Ada 2012 allows the programmer to include pre- and postconditions on subprograms, as
well as other assertions, that encode higher level correctness properties (e.g., a sort
procedure produces a sorted permutation of its input). Again, GNATprove generates verification
conditions that, if discharged, will statically show those properties will always hold.

At the time of this writing the \SPARK\ tools ship with two back-end theorem provers, Alt-Ergo
\cite{alt-ergo} and CVC4 \cite{barrett2011}. Two provers are used to take advantage of their
complementary strengths; verification conditions unprovable by one prover might be handled by
the other. It is possible to configure the \SPARK\ tools to use only one prover or additional
provers obtained separately, such as Microsoft's Z3 \cite{Moura2008}.

The GPS integrated development environment developed by AdaCore provides a convenient front-end
to the \SPARK\ tools. Using the tools can be as easy as selecting ``Prove File'' from the GPS
menus. The result is a list of locations where unproved verification conditions exist, if any.
The programmer can then view and edit those locations as necessary.

Proofs fail for three reasons:
\begin{itemize}
\item The code is incorrect. The check being analyzed might actually fail.
\item The theorem prover(s) are not powerful enough to complete the proofs.
\item There is insufficient information in the program to complete the proofs.
\end{itemize}

Most of the skill in using the \SPARK\ tools is in determining which of these cases is the
problem, and in modifying the program to deal with that situation.

It is important to understand that the GNAT Ada compiler can insert runtime checks for all the
\SPARK\ assertions as well as the Ada language mandated checks. During testing it would be
typical to build the program with these runtime checks enabled. Thus checks that can't be
completely proved can still be tested. Once all checks are proved, the runtime checking can be
disabled, saving both space and time in the final program without compromising safety.

\subsection{Language}
\label{sec:spark2014-language}

The \SPARK\ language is a subset of Ada in that certain Ada features that are difficult to
analyze using current technology have been removed from the language. Specifically \SPARK\
supports neither exception handling nor access types (pointers). In \SPARK\ it is necessary to
report errors using returned status values. However, \SPARK's flow analysis ensures that all
such values are checked. It is not possible to ignore error codes in a \SPARK\ program that
passes examination without warning.

The lack of access types may seem more limiting but Ada, in general, requires less use of
explicit indirection than is typical in C programs. In Ada, and in \SPARK, arrays are first
class citizens of the language and can be passed into and returned from subprograms directly.
Also arrays can be dynamically sized on the stack without the use of an explicit memory
allocator.

The \SPARK\ language also extends Ada with additional aspects that enrich declarations and
additional assertions that describe conditions that must hold true in every execution of the
program. The additional aspects include data dependency and information dependency declarations.
The additional assertions include pre- and postconditions, loop invariants, subtype predicates,
and other related things.

As an example consider the following specification of a \SPARK\ package containing a single
global datebook object along with subprograms for manipulating it:

\begin{lstlisting}
with Dates;
use type Dates.Datetime;

package Datebook
  with
    SPARK_Mode => On,
    Abstract_State => State
is
   Maximum_Number_Of_Events : constant := 64;
   subtype Event_Count_Type is Natural range 0 .. Maximum_Number_Of_Events;

   type Status_Type is (Success, Description_Too_Long, Insufficient_Space, No_Event);

   -- Initializes the datebook.
   procedure Initialize
   with
     Global => (Output => State),
     Depends => (State => null);

   -- Adds an event to the datebook.
   procedure Add_Event
     (Description : in  String;
      Date        : in  Dates.Datetime;
      Status      : out Status_Type)
   with
     Global => (In_Out => State),
     Depends => (State =>+ (Description, Date), Status => (Description, State));

   -- Other subprograms as required...

end Datebook;
\end{lstlisting}

The package is decorated with a |SPARK_Mode| aspect set to |On| indicating that this compilation
unit is intended to abide by the restrictions of the \SPARK\ language. The fact that the package
contains internal global state is declared explicitly using the |Abstract_State| aspect. How
that internal state is manipulated by the subprograms is also declared explicitly using the
|Global| and |Depends| aspects. For example, the |Add_Event| procedure both reads and writes the
global state. Specifically the new state depends on itself (the meaning of the plus sign in the
=>+ notation) and on the |Description| and |Data| parameters.

The \SPARK\ tools use this information to verify that all values are initialized before use and
that all computed results are used in some way. For example, calling |Add_Event| before calling
|Initialize| is detected because |Add_Event| reads the package state and thus requires it to be
initialized first. Similarly since |Status| is an out parameter of the procedure the \SPARK\
tools will verify that its value is used in some way; ignoring status codes is not allowed.

The \SPARK\ tools will further verify that the dependency declarations are supported by the
implementation in the package body (not shown here for the sake of brevity).

As another example consider the following specification of a \SPARK\ package containing a search
procedure:

\begin{lstlisting}
package Searchers
  with SPARK_Mode => On
is
   subtype Index_Type is Positive range 1 .. 100;
   type Array_Type is array(Index_Type) of Integer;

   procedure Binary_Search (Search_Item : in  Integer;
                            Items       : in  Array_Type;
                            Found       : out Boolean;
                            Result      : out Index_Type)
      with
         Pre =>
            (for all J in Items'Range =>
               (for all K in J + 1 .. Items'Last => Items(J) <= Items(K))),
         Post =>
           (if Found then Search_Item = Items(Result)
                     else (for all J in Items'Range => Search_Item /= Items(J)));

end Searchers3;
\end{lstlisting}

Following normal Ada style, an array type is defined that is indexed over a subrange of the
range of positive integers. The |Binary_Search| procedure takes an item to search for, an array
to search, and outputs a Boolean flag to indicate if the item is found along with the item's
location in the array if it is.

The procedure declaration is enhanced with additional semantic information in the form of pre-
and postconditions. The precondition states that the input array is sorted. The postcondition
states that if the item is found the returned index is, in fact, the location of the item. On
the other hand if the item is not found, it does not exist in the array.

The body of this package showing the implementation of the procedure is:

\begin{lstlisting}
package body Searchers
  with SPARK_Mode => On
is
   procedure Binary_Search (Search_Item : in  Integer;
                            Items       : in  Array_Type;
                            Found       : out Boolean;
                            Result      : out Index_Type) is
      Low_Index  : Index_Type := Items'First;
      Mid_Index  : Index_Type;
      High_Index : Index_Type := Items'Last;
   begin
      Found  := False;
      Result := Items'First;  -- Initialize Result to "not found" case.

      -- If the item is out of range, it is not found.
      if Search_Item < Items(Low_Index) or Items(High_Index) < Search_Item then
         return;
      end if;

      loop
         Mid_Index := (Low_Index + High_Index) / 2;
         if Search_Item = Items(Mid_Index) then
            Found  := True;
            Result := Mid_Index;
            return;
         end if;

         exit when Low_Index = High_Index;

         pragma Loop_Invariant (not Found);
         pragma Loop_Invariant (Mid_Index in Low_Index .. High_Index - 1);
         pragma Loop_Invariant (Items(Low_Index) <= Search_Item);
         pragma Loop_Invariant (Search_Item <= Items(High_Index));
         pragma Loop_Variant (Decreases => High_Index - Low_Index);

         if Items(Mid_Index) < Search_Item then
            if Search_Item < Items(Mid_Index + 1) then
               return;
            end if;
            Low_Index := Mid_Index + 1;
         else
            High_Index := Mid_Index;
         end if;

      end loop;
   end Binary_Search;

end Searchers;
\end{lstlisting}

The \SPARK\ tools will first generate verification conditions at each place in the body where an
Ada check is required. For example every place where the |Items| array is accessed must be
checked to ensure the index used is in range. Using the precondition as an initial hypotheses,
and adding information based on the actions taken in the procedure, the \SPARK\ tools will
generate a verification condition to show that the postcondition is always true. Furthermore at
every call site a verification condition will be generated to show that the precondition must be
true at that call site.

In this example all of these verification conditions are proved automatically showing that the
procedure is free of unexpected runtime errors \emph{and} that it always honors its strong
postcondition (given the precondition).

The |Loop_Invariant| pragmas in the procedure where written to assist the proving process. They
represent conditions that must be true at that point for every iteration of the enclosing loop.
The \SPARK\ tools prove that the invariants are true on the first iteration and that they remain
true on all following iterations. The tools can then use the conditions in the invariants to
complete following proofs, such as the postcondition in this case.

The |Loop_Variant| pragma is used to prove that the loop will eventually terminate. It gives an
expression that, in this case, always decreases with each loop iteration. Because the types
involved are bounded and because the \SPARK\ tools have already proved that overflow errors are
impossible, even in the assertion expressions themselves, it follows that the loop must end
since the value of a bounded expression can't decrease forever.

Although this example can only search arrays of 100 integers, it is possible, although
admittedly more difficult, to write general purpose code that is similarly proved free of
errors. Overall these examples only give a flavor of \SPARK\ and many features and details have
been left out for the sake of brevity.

\lstDeleteShortInline|


\section{New Work}
\label{sec:new}


\section{Conclusion}
\label{sec:conclusion}

XDR2OS3 is wonderful. Using it will change your life forever.


\bibliographystyle{plain}
\bibliography{../references-CubeSat,../references-SPARK}

\end{document}
