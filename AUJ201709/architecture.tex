
\section{CubedOS Architecture}
\label{section-architecture}

To understand the context of the CubedOS architecture, it is useful to compare the architecture
of a CubedOS application with that of a more traditional application. Since CubedOS is written
in \SPARK\ and must abide by the restrictions of Ravenscar, we compare CubedOS with other
Ravenscar-based approaches.

Figure~\ref{fig:traditional-architecture} shows an example application using Ravenscar tasking.
Tasks, which must all be library level infinite loops, are shown as open circles and labeled as
$T_1$ through $T_4$. Tasks communicate with each other via protected objects, shown as solid
circles and labeled as $PO_1$ through $PO_4$.

\begin{figure}[tbhp]
  \center
  \scalebox{0.40}{\includegraphics*{Figure-Traditional.pdf}}
  \caption{Traditional Ravenscar-based Architecture}
  \label{fig:traditional-architecture}
\end{figure}

Arrows from a sending task to a protected object indicate calls to a protected procedure to
install information in the protected object. Arrows from a protected object to a receiving task
indicate calls to an entry in the protected object used to pick up information previously stored
in the object. Entry calls will block if no information is yet available but protected procedure
calls do not block.

Ravenscar requires that protected objects have at most one entry and that at most one task can
be queued on that entry. In CubedOS applications each protected object is serviced by exactly
one task. This ensures that two tasks will never accidentally be queued on the protected
object's entry. In the figure this means only one arrow can emanate from a protected object.
However, multiple arrows can lead to a protected object, since it is permitted for many tasks to
call the same protected procedure or for there to be multiple protected procedures in a given
protected object.

In the example application of Figure~\ref{fig:traditional-architecture}, tasks $T_1$ and $T_3$
call protected procedures in two different protected objects. This presents no problems since
protected procedures never block, allowing a task to call both procedures in a timely manner.
However, task $T_3$ calls two entries, one in $PO_1$ and another in $PO_2$. Since entry calls
can block, this means the task might get suspended on one of the calls leaving the other
protected object without service for an extended time. The application needs to either be
written so that will never happen or be such that it doesn't matter if it does.

There are several advantages over the traditional organization:

\begin{itemize}
\item The protected objects can be tuned to transmit only the information needed so the overhead
  can be kept minimal.
\item The parameters of the protected procedures and entries specify the precise types of the
  data transfered so compile-time type safety is provided.
\item The communication patterns of the application are known statically, facilitating analysis.
\end{itemize}

However the traditional architecture also includes some disadvantages:

\begin{itemize}
\item The protected objects must all be custom designed and individually implemented, creating a
  burden for the application developer.
\item The communication patterns are relatively inflexible. Changing them requires overhauling
  the application.
\end{itemize}

A CubedOS application has an architecture as shown in Figure~\ref{fig:cubedos-architecture}. In
this case CubedOS provides the communication infrastructure as an array of general purpose,
protected mailbox objects. CubedOS modules communicate by sending messages to the receiver
module's mailbox. The messages are unstructured octet streams, and thus completely generic. Each
active module has exactly one mailbox associated with it and contains a task dedicated to
servicing that mailbox. That task extracts messages from the mailbox, and then decodes and acts
on each message. Active CubedOS modules can also contain internal tasks as part of their
implementation, but those tasks do not participate in message processing (although they can send
messages) and are not important here.

\begin{figure}[tbhp]
  \center
  \scalebox{0.40}{\includegraphics*{Figure-CubedOS.pdf}}
  \caption{CubedOS-based Architecture}
  \label{fig:cubedos-architecture}
\end{figure}

The communication connections shown in Figure~\ref{fig:cubedos-architecture} are the same as
those shown in Figure~\ref{fig:traditional-architecture} except that the two communication paths
from $T_1$ to $T_3$ are combined into a single path going through one mailbox.

CubedOS relieves the application developer of the problem of creating the communications
infrastructure manually. Adding new message types is simplified with the help of a tool,
XDR2OS3, that we describe in Section~\ref{section-message-encoding}. In addition to providing
basic, bounded mailboxes, CubedOS also provides other services such as message priorities and
multiple sending modalities (for example, best effort versus guaranteed delivery). Many of these
additional services would be tedious to provide on a case-by-case basis following the
traditional architecture. CubedOS also allows any module to potentially send a message to any
other module. Thus the communication paths in the running application are very flexible and
dynamic.

Although the CubedOS architecture supports only point-to-point message passing, the CubedOS
system provides an active module supporting a publish/subscribe discipline. The module allows
multiple channels to be created to which other modules can subscribe. Publisher modules can then
send messages to one or more channels, allowing for message broadcast and multicast. Since the
messages themselves are unstructured octet streams, the publish/subscribe module can handle them
generically without being modified to account for new message types.

Every CubedOS module has a statically assigned ID number. Messages sent from a module include
the ID number of the sender. This allows a server module to return reply messages without
statically knowing its clients. Thus server modules can be written as part of a general purpose
``module library'' and used without modification in a variety of applications. We have started
compiling a registry of ``well known'' module IDs for common services, such as file handling and
timer services. This allows CubedOS module libraries to make use of well known services and
remain reusable. Here active CubedOS modules resemble network clients and servers where the
module IDs play the role of a network address. Extending the architecture across different
physical machines, or between different operating system processes on the same machine, is an
interesting area for future work.

We are also defining standard message interfaces to certain services, such as file handling,
that third party modules could implement. This allows modules to use a service without knowing
which specific implementation backs that service.

However, CubedOS's architecture also carries some significant disadvantages as well:

\begin{itemize}
\item All mailboxes must have the same size since they are stored in an array. Some mailboxes
  will be larger than necessary, wasting space.

\item All messages must have the same type and thus the same size. Some messages will be larger
  than necessary and slower to copy than necessary.

  The common message type also requires that typed information sent from one module to another
  be encoded into a raw octet format when sent, and decoded back into specifically typed data
  when received. This encoding and decoding increases the runtime overhead of message passing
  and reduces static type safety. Modules must defend themselves, at runtime, from malformed or
  inappropriate messages, causing certain errors that were compile-time errors in the
  traditional architecture to now be runtime errors. This is exactly counter to the general
  goals of high integrity system development.

\item In order to return reply messages, the mailboxes must be addressable at runtime using
  module ID numbers. Accessing a statically named mailbox isn't general enough. As a result, the
  precise communication paths used by the system cannot easily be determined statically.

  In particular, since \SPARK\ does not attempt to track information flow through individual
  array elements, it is necessary for us to manually justify certain \SPARK\ flow messages. The
  architecture of CubedOS ensures that there is a one-to-one correspondence between a module and
  its mailbox. The tools don't know this, and the spurious flow messages they produce must be
  suppressed.
\end{itemize}

The details of CubedOS mitigate, to some degree, the problems above. For example, the mailbox
array is actually instantiated from a generic unit by the application developer. This allows the
developer to tune the sizes of the mailboxes, and the messages they contain, to the
application's needs. CubedOS does not attempt to provide a one-size-fits-all mailbox array that
will be satisfactory to all applications.

Also every well behaved CubedOS module should contain an |API| package with subprograms for
encoding and decoding messages. This package is generated by the XDR2OS3 tool that we describe
in Section~\ref{section-message-encoding}. The parameters to these subprograms correspond to the
parameters of the protected procedures and entries in the traditional architecture, and provide
much of the same type safety. However, using the API subprograms is not enforced by the
compiler. It is also possible to accidentally send a message to the wrong mailbox. Thus modules
still need to include runtime error checking to detect and handle these problems.

So far we have described two extremes: a traditional approach that does not use CubedOS at all,
and an approach that entirely relies on CubedOS. However, hybrid approaches are also possible.
Figure~\ref{fig:hybrid-architecture} shows a combination of several CubedOS mailboxes and a
hand-made, optimized protected object to mediate communication from $T_3$ to $T_4$.

\begin{figure}[tbhp]
  \center
  \scalebox{0.40}{\includegraphics*{Figure-Hybrid.pdf}}
  \caption{Hybrid Architecture}
  \label{fig:hybrid-architecture}
\end{figure}

This provides the best of both worlds. The simplicity and flexibility of CubedOS can be used
where it makes sense to do so, and yet critical communications can still be optimized if the
results of profiling indicate a need. In Figure~\ref{fig:hybrid-architecture} task $T_4$ can't
be reached by CubedOS messages. The hand-made protected object creates a degree of isolation
that can also simplify analysis as compared to a pure CubedOS system.

It is also possible to instantiate the CubedOS message manager multiple times in the same
application, effectively creating multiple communication domains using separate mailbox arrays.
Figure~\ref{fig:multi-domain} shows an example of where $T_4$ is in a separate domain from the
other modules (because it receives from a mailbox that is separate from the others).

\begin{figure}[tbhp]
  \center
  \scalebox{0.40}{\includegraphics*{Figure-MultiDomain.pdf}}
  \caption{Multiple Communication Domains}
  \label{fig:multi-domain}
\end{figure}

This approach allows the CubedOS infrastructure to be used for easy development while still
partitioning the system into semi-independent sections. For example, the sizes of the mailboxes
and of the messages used in each communication domain need not be the same. The parts of the
application that require large messages could be grouped into a domain separate from the parts
that only require small messages.

Notice in Figure~\ref{fig:multi-domain} tasks (modules) $T_3$ and $T_4$ send messages into
multiple domains. This is, of course, sometimes necessary if the domains are going to interact.
Modules that do this will need multiple module ID values scoped to different domains. At the
moment the handling of this is largely a matter of manual configuration, which is reasonable for
the relatively small programs typical of CubeSat missions. Creating a more comprehensive
solution for module and domain addresses would be necessary as part of extending the
architecture to multiple processes or machines as mentioned earlier. It is also likely that a
naming service of some kind would need to be added to the module library provided by CubedOS.
This is also an area for future work.
